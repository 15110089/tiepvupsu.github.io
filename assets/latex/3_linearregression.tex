%!TEX root = ML_book.tex
\chapter{Linear Regression}
\label{cha:linearregression}
\section{Giới thiệu}
\label{sec:linearregression_-gioi-thieu}
% <a name="-gioi-thieu"></a>

Quay lại [ví dụ đơn giản được nêu trong bài trước]%(/2016/12/27/categories/#regression): một căn nhà rộng $x_1 ~ \text{m}^2$, có $x_2$ phòng ngủ và cách trung tâm thành phố $x_3~ \text{km}$ có giá là bao nhiêu. Giả sử chúng ta đã có số liệu thống kê từ 1000 căn nhà trong thành phố đó, liệu rằng khi có một căn nhà mới với các thông số về diện tích, số phòng ngủ và khoảng cách tới trung tâm, chúng ta có thể dự đoán được giá của căn nhà đó không? Nếu có thì hàm dự đoán $y = f(\mathbf{x}) $ sẽ có dạng như thế nào. Ở đây $\mathbf{x} = [x_1, x_2, x_3] $ là một vector hàng chứa thông tin _input_, $y$ là một số vô hướng (scalar) biểu diễn _output_ (tức giá của căn nhà trong ví dụ này).

\textbf{Lưu ý về ký hiệu toán học:} \textit{trong các bài viết của tôi, các số vô hướng được biểu diễn bởi các chữ cái viết ở dạng không in đậm, có thể viết hoa, ví dụ $x_1, N, y, k$. Các vector được biểu diễn bằng các chữ cái thường in đậm, ví dụ $\mathbf{y}, \mathbf{x}_1 $. Các ma trận được biểu diễn bởi các chữ viết hoa in đậm, ví dụ $\mathbf{X, Y, W} $.}

Một cách đơn giản nhất, chúng ta có thể thấy rằng: i) diện tích nhà càng lớn thì giá nhà càng cao; ii) số lượng phòng ngủ càng lớn thì giá nhà càng cao; iii) càng xa trung tâm thì giá nhà càng giảm. Một hàm số đơn giản nhất có thể mô tả mối quan hệ giữa giá nhà và 3 đại lượng đầu vào là: 